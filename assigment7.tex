\documentclass{article} 


\usepackage{pawlowski}
\usepackage{amsmath}
\usepackage{natbib}
\usepackage{graphicx}
\usepackage{indentfirst} 

\title{Assignment 7}
\author{Trystin Wheatcroft}

\begin{document}

\section{Numerical Derivatives}

\subsection*{Preliminary: Taylor Expansion}
The Taylor expansion is a method of taking numerical derivatives that takes the form: 
$$f(x) =\sum_{n=0}^{\infty}f^n(x_0)*(x-x_0)^n/n!$$
This is an infinite summation of terms that approximates an unknown point in a function using a known point. Using this expansion, we can derive two schemes for numerical differentiation; the Central Differencing scheme (\ref{Central Differencing}) and a Third Order Scheme.

\subsection*{Central Differencing}
Starting with the Taylor Expansion to the $1^{st}$ order, we can derive an equation for solving the first derivative of a function in two ways. First using a point that is beyond the point we're trying to find $(f(x+h))$ and by using a point behind the point we're trying to find $(f(x-h))$. These two methods are called Forward Differencing (\ref{Forward Differencing}) and Backward Differencing (\ref{Backward Differencing}). They are derived by:
\begin{align}
    \label{Forward Differencing}
    &Forward \nonumber \\
    f(x+h) &= f(x) + f'(x)*h \nonumber \\
    f'(x) &= \frac{f(x+h) -f(x)}{h} \\
    \nonumber \\
    \label{Backward Differencing}  
    &Backward \nonumber \\
    f(x-h) &= f(x) - f'(x)*h \nonumber \\
    f'(x) &= \frac{f(x) - f(x+h)}{h}
\end{align}
By combining these two methods, we get the Central Differencing scheme (\ref{Central Differencing}), which no longer relies on f(x):
\begin{align}
    \label{Central Differencing}
    f'(x) = (f(x+h) - f(x-h)) * 2h
\end{align}

\subsection*{Third Order Scheme}
There are also methods that start with the Taylor Series to the $3^{rd}$ order, following similar methods as above, we can derive three-point forward, backward, and central differencing formulas:

\begin{align}
    \label{3pt Forward Differencing}
    &Forward \nonumber \\
    f'(x) &= \frac{-3f(x) +4f(x+h) - f(x+2h)}{2h}+O(h^2)    \\
    \label{3pt Backward Differencing} 
    &Backward \nonumber \\
    f'(x) &= \frac{-f(x+2h) +8f(x+h) - 8f(x-h)+f(x-2h)}{12h}+O(h^4)   \\ 
    \label{3pt Central Differencing} 
    &Central \nonumber \\
    f'(x) &= \frac{f(x-2h)-4f(x-h)+3f(x)}{2h} = O(h^2)
\end{align}

\newpage

\section{Electric Potential of Many Charges}

\subsection{Summary}
This assignment was to rework our former electric potential and electric field modeling program to make it work for any number of charges, defined in a given file titled creategrid.py.

\begin{figure}
    \centering
    \includegraphics[width=0.5\linewidth]{potential.png}
    \caption{Electric Potential}
    \label{fig:potential}
\end{figure}

\begin{figure}
    \centering
    \includegraphics[width=0.5\linewidth]{electricfield.png}
    \caption{Electric Field}
    \label{fig:field}
\end{figure}

\newpage

\section{Harmonic Oscillator}

\subsection{Summary}
When $C^2$ is 0, there is no damping, so the oscillator will oscillate forever without loss of energy \ref{fig:C=0}. When the oscillator is critically damped, The system returns to equilibrium in the shortest possible time without oscillating \ref{fig:critdamp}. When under damped, the motion still oscillates while the amplitude decays \ref{fig:undamp}. When overdamped, the system slowly returns to equilibrium without oscillating \ref{fig:overdamp}.

\begin{figure}[bp]
    \centering
    \includegraphics[width=0.5\linewidth]{Figure 2025-12-03 C0.png}
    \caption{C=0}
    \label{fig:C=0}
\end{figure}

\begin{figure}[bp]
    \centering
    \includegraphics[width=0.5\linewidth]{Figure 2025-12-03 critdamp.png}
    \caption{Critically Damped}
    \label{fig:critdamp}
\end{figure}

\begin{figure}
    \centering
    \includegraphics[width=0.5\linewidth]{Figure 2025-12-04 overdamped.png}
    \caption{Overdamped}
    \label{fig:overdamp}
\end{figure}

\begin{figure}
    \centering
    \includegraphics[width=0.5\linewidth]{Figure 2025-12-04 underdamped.png}
    \caption{Under damped}
    \label{fig:undamp}
\end{figure}


\newpage
\section{Cycling Without Drag}

\subsection{Summary}
For this section I used Euler's method to solve for the velocity of an ideal cyclist without drag or other degenerative forces.

\begin{figure}[bp]
    \centering
    \includegraphics[width=0.5\linewidth]{bicycle.png}
    \caption{Velocity vs Time}
    \label{fig:bicycle}
\end{figure}




\end{document}
